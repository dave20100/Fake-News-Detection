\chapter*{Wstęp}
\section*{Motywacja}
Ogromna w ostatnich latach popularność różnego rodzaju mediów społecznościowych, a co za tym 
idzie wymiany ogromnej ilości informacji między ludźmi powoduje, że tak jak nigdy wcześniej 
w historii spotkać się można z masową dezinformacją i skutkami jakie może ona ze sobą nieść.
Nieprawdziwe informacje stanowią realne zagrożenie dla każdego z nas dlatego odnalezienie 
prostego sposobu na ich wykrycie staje się coraz ważniejsze i mogłoby pozwolić na 
zredukowanie lub całkowite wyeliminowanie szansy na bycie oszukanym podczas zbierania
informacji z różnych stron internetowych. Popularne ostatnimi czasy algorytmy uczenia 
maszynowego mogą stanowić rozwiązanie tego problemu.
\section*{Cel pracy}
Celem pracy jest sprawdzenie efektywności popularnych algorytmów uczenia maszynowego w rozwiązywaniu 
zadania klasyfikacji tekstów na prawdziwe i nieprawdziwe informacje. Badana metoda opiera 
się na analizie tekstu podzielonego na tak zwane NGramy czyli sekwencje jednego lub kilku znaków.
Metoda ta ma jest stosowana w detekcji spamu w różnego rodzaju skrzynkach mailowych.
W badaniu sprawdzano jak zmiana długości NGramów wpływa na efektywność algorytmów czas ich 
uczenia oraz czas wykonywania przez nie predykcji. 
\section*{Zawartość pracy}
Praca składa się z sześciu rodziałów pierwsze trzy zawierają teoretyczne wprowadzenie 
do tematów związanych z celem pracy. Kolejne dwa rozdziały składają się z opisu metod jakie 
zostały wykorzystane do osiągnięcia celu oraz analizę wykonanych badań. Ostatni rozdział 
stanowi podsumowanie całej pracy.

Dokładny opis zawartości każdego z rozdziałów wygląda następująco:
\begin{enumerate}
    \item W pierwszym rozdziale krótko została opisana historia nieprawdziwych informacji 
    i ich rozprzestrzeniania. Zdefiniowano pojęcie fake news i omówiono jego znaczenie w
    dzisiejszym świecie. Omówiono także sposoby ochrony przed nieprawdziwymi informacjami.
    Opisano także zagrożenie jakie stanowi stworzona w ostatnich latach technologia o 
    nazwie deepfake.
    \item Rozdział drugi zawiera opis technologii jaką jest uczenie maszynowe. Zostały
    omówione jedne z popularniejszych algorytmów uczenia maszynowego, które wykorzystano
    do wykonania badań. Został także uwzględniony opis zagrożenia jakie może stanowić uczenie
    maszynowe dla ludzkości.
    \item Trzeci rozdział krótko opisuje dziedzinę informatyki, którą jest przetwarzanie 
    języków naturalnych. Zostały opisane przykłady jej zastosowania w dzisiejszym świecie.
    Krótko omówione zostały także sposoby przygotowywania danych tekstowych. Ostatnim elementem
    rozdziału jest opis metod wektoryzacji czyli zamiany tekstu na formę numeryczną. 
    \item W rozdziale czwartym zawarty jest opis systemu, który został 
    stworzony w celu wykonania badań. Opisano wykorzystane w nim technologie oraz motywację 
    wyboru każdego z nich. Wymieniono wymagania funkcjonalne, które musi spełniać ten system oraz
    zawarto krótki opis jego implementacji.
    \item Piąty rozdział zawiera ocenę eksperymentalną wykonanych badań. Opisane są w nim 
    warunki przeprowadzonego eksperymentu wraz ze specyfikacją danych, wyniki badań a także 
    analiza wyników. Na zakończenie rozdziału sformułowano wnioski wynikające z wyników badań.  
\end{enumerate}