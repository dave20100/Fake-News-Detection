\chapter*{Wstęp}
\section*{Motywacja}
Ogromna w ostatnich latach popularność różnego rodzaju mediów społecznościowych, a co za tym 
idzie wymiana ogromnej ilości informacji między ludźmi powoduje, że tak jak nigdy wcześniej 
w historii spotkać się można z masową dezinformacją i skutkami jakie może ona ze sobą nieść ~\cite{fakenewsreach}.
Nieprawdziwe informacje stanowią realne zagrożenie dla każdego z nas, dlatego odnalezienie 
prostego sposobu na ich wykrycie staje się coraz ważniejsze i mogłoby 
zredukować lub całkowicie wyeliminować szansę na bycie oszukanym podczas pozyskiwania
informacji z różnych stron internetowych. Popularne ostatnimi czasy algorytmy uczenia 
maszynowego mogą stanowić rozwiązanie tego problemu.
\section*{Cel pracy}
Celem pracy jest sprawdzenie efektywności popularnych algorytmów uczenia maszynowego w rozwiązywaniu 
zadania klasyfikacji nieprawdziwych informacji. Badana metoda opiera 
się na analizie tekstu podzielonego na tak zwane ngramy czyli sekwencje jednego lub kilku znaków.
Metoda ta jest stosowana podczas detekcji spamu w różnego rodzaju skrzynkach mailowych.
W badaniu sprawdzano jak zmiana długości NGramów wpływa na poprawność algorytmów, czas ich 
uczenia oraz czas wykonywania przez nie predykcji. 
\section*{Zawartość pracy}
Praca składa się z sześciu rozdziałów. Pierwsze trzy rozdziały zawierają teoretyczne wprowadzenie 
do tematów związanych z celem pracy. Kolejne dwa rozdziały składają się z opisu metod jakie 
zostały wykorzystane do osiągnięcia celu oraz analizę wykonanych badań. Ostatni rozdział 
stanowi podsumowanie całej pracy.

Dokładny opis zawartości każdego z rozdziałów wygląda następująco:
\begin{itemize}
    \item Pierwszy rozdział zawiera krótki opis historii nieprawdziwych informacji
    i ich rozprzestrzeniania, a także definicję pojęcia Fake News i omówienie jego znaczenia 
    w dzisiejszym świecie. Zostały w nim również opisane sposoby ochrony przed nieprawdziwymi informacjami
    oraz zagrożenia jakie stanowi stworzona w ostatnich latach technologia deepfake.
    \item Rozdział drugi zawiera opis technologii jaką jest uczenie maszynowe. Zostały
    omówione jedne z popularniejszych algorytmów uczenia maszynowego, które wykorzystano
    do wykonania badań. Został także uwzględniony opis zagrożenia jakie może stanowić uczenie
    maszynowe dla ludzkości.
    \item Trzeci rozdział krótko opisuje dziedzinę informatyki, którą jest przetwarzanie 
    języków naturalnych. Zostały opisane przykłady jej zastosowania w dzisiejszym świecie.
    W rozdziale znajduje się również omówienie sposobów przygotowywania danych tekstowych. Ostatnim elementem
    rozdziału jest opis metod wektoryzacji, czyli zamiany tekstu na formę numeryczną. 
    \item W rozdziale czwartym zawarty jest opis systemu, który został 
    stworzony w celu wykonania badań. Opisuje wykorzystane w nim technologie oraz motywację 
    wyboru każdego z nich. Zawarte są w nim wymagania funkcjonalne, które musi spełniać ten system oraz
    krótki opis jego implementacji.
    \item Piąty rozdział zawiera ocenę eksperymentalną wykonanych badań. Opisane są w nim 
    warunki przeprowadzonego eksperymentu wraz ze specyfikacją danych, wyniki badań a także 
    analiza wyników. Zakończenie rozdziału zawiera wnioski wynikające z analizy badań.  
\end{itemize}