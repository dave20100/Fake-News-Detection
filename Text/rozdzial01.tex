\chapter*{Wstęp}
\section*{Motywacja}
Ogromna ilość informacji, która codziennie pojawia się przed oczyma każdego użytkownika internetu
powoduje że
\section*{Cel pracy}
Celem pracy jest sprawdzenie efektywności algorytmów uczenia maszynowego 
\section*{Zawartość pracy}
Praca składa się z sześciu rodziałów pierwsze trzy zawierają teoretyczne wprowadzenie 
do tematów związanych z celem pracy. Kolejne dwa rozdziały składają się z opisu metod jakie 
zostały wykorzystane do osiągnięcia celu oraz analizę wykonanych badań. Ostatni rozdział 
stanowi podsumowanie całej pracy.

Dokładny opis zawartości każdego z rozdziałów wygląda następująco:
\begin{enumerate}
    \item W pierwszym rozdziale krótko została opisana historia nieprawdziwych informacji 
    i ich rozprzestrzeniania. Zdefiniowano pojęcie fake news i omówiono jego znaczenie w
    dzisiejszym świecie. Omówiono także sposoby ochrony przed nieprawdziwymi informacjami.
    Opisano także jakie zagrożenie stanowi stworzona w ostatnich latach aplikacja o 
    nazwie deepfake.
    \item Rozdział drugi zawiera opis technologii jaką jest uczenie maszynowe. Zostały
    omówione jedne z popularniejszych algorytmów uczenia maszynowego, które wykorzystano
    do wykonania badań. Został także uwzględniony opis zagrożenia jakie może stanowić uczenie
    maszynowe dla ludzkości.
    \item Trzeci rozdział krótko opisuje dziedzinę informatyki, którą jest przetwarzanie 
    języków naturalnych. Zostały opisane przykłady jej zastosowania w dzisiejszym świecie.
    Krótko omówione zostały także sposoby przygotowywania danych tekstowych. Ostatnim elementem
    rozdziału jest opis metod wektoryzacji czyli zamiany tekstu na formę numeryczną. 
    \item Czwarty
    \item Piąty
\end{enumerate}