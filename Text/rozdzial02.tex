\chapter{Informacje nieprawdziwe w dobie Internetu}
Pojęcie \textit{fake news} odnosi się do 
informacji, które pomimo że nie posiadają pokrycia z rzeczywistością są
przedstawiane jako prawdziwe w mediach takich jak np. wiadomości, artykuły, 
portale społecznościowe itd.
Zwrot ten jest neologizmem i w języku angielskim oznacza dosłownie ``Fałszywe wiadomości''.
Tego rodzaju wiadomości często wykorzystywane są w tworzeniu treści humorystycznych 
takich jak satyra, jednak ich główne przeznaczenie sprowadza się do spełniania jednego z dwóch zadań:
\begin{itemize}
    \item oszukać odbiorcę i wpłynąć na jego poglądy w sposób żądany przez autora danej informacji (propaganda),
    \item namówienie go na zakup czegoś czego w innym przypadku by on nie kupił (reklama).
\end{itemize} 
Niektóre fałszywe informacje łączą oba cele.

\textit{Fake newsy} stały się bardzo popularnym zagadnieniem w ostatnich czasach ponieważ
internet, a w szczególności media społecznościowe dają możliwość przekazywania
informacji z niespotykaną wcześniej prędkością dzięki czemu rozprzestrzenianie 
dezinformacji stało się zadaniem stosunkowo prostym.

Zagadnienie to zyskało ogromny rozgłos podczas kampanii wyborczej oraz
prezydentury Donalda Trumpa, który zasłynął z częstego wykorzystywania 
tego zwrotu podczas wywiadów, debat oraz wypowiedzi w mediach społecznościowych
takich jak Twitter.
Do roku 2020 roku pojęcie \textit{fake news} zostało umieszczone w słownikach języka angielskiego
takich jak ``Oxford English Dictionary'', ``Macmillan Dictionary''.
\begin{figure}[h!]
    \centering
    \includegraphics[width=0.7\textwidth]{./Img/Trump-Fake-News.png}
    \caption{Post udostępniony przez Donalda Trumpa na portalu Twitter Źródło: https://twitter.com/}
\end{figure}

Według projektu ``First Draft News'' założonego przez dziewięć organizacji, 
w skład których wchodzą Google, Facebook oraz Twitter 
możemy wyróżnić siedem typów \textit{fake newsów}~\cite{TypesOfFakeNews}:
\begin{itemize}
    \item satyra bądź parodia - nie ma na celu wyrządzić krzywdy ale może oszukać,
    \item fałszywe połączenie - nagłówki oraz obrazy nie mają powiązania z zawartością,
    \item myląca zawartość - tekst napisany w mylący sposób,
    \item fałszywy kontekst - prawdziwa zawartość powiązana ze złym kontekstem,
    \item oszukana zawartość - źródła pochodzenia informacji są fałszywe,
    \item zmanipulowana zawartość - prawdziwa zawartość zmanipulowana w odpowiedni sposób by oszukać odbiorcę,
    \item sfabrykowana zawartość - całkowicie zmyślona zawartość mająca na celu wyrządzić krzywdę.
\end{itemize}
Jak podaje słownik ``Merriam Webster'' po raz pierwszy wykorzystano zwrot
\textit{fake news} w roku 1890. Wiele nieprawdziwych wiadomości wzbudziło tak
wielkie zainteresowanie, że na zawsze zapisały się one na kartach historii. Przykładami takich fake newsów są~\cite{examplesoffakenews}:
\begin{itemize}
    \item życie na księżycu - w roku 1835 nowojorska gazeta The Sun opublikowała 6 artykułów, które ze szczegółami
    opisywały odkrycie życia na księżycu. Odkrywcą według artykułu miał być znany astronauta John Herschel.
    Gazeta przyciagnęła ogromne zainteresowanie co znacznie podniosło jej popularność, w tym samym roku przyznała się ona do 
    nieprawdziwości artykułów,
    \item  Kuba Rozpruwacz -w roku 1888 w Londynie popełniono serię brutalnych zabójstw, których popełnienie przypisano 
    seryjnemu mordercy o pseudonimie Kuba Rozpruwacz. Wiele gazet w tamtym okresie wykorzystywało historię o zabójstwach 
    w celu przyciągnięcia uwagi przechodniów, nie zawsze przekazując prawdziwe informacje. Jednym z przykładów była para 
    sprzedawców gazet William Macdonald oraz George Write, którzy sprzedając swoje artykuły wykrzykiwali o aresztowaniu
    Kuby Rozpruwacza, informacja ta była jednak całkowicie fałszywa, ponieważ morderca nie został nigdy odnaleziony.
\end{itemize}


Jednym z najsłynniejszych działań wykorzystujących \textit{fake newsy} w złych celach jest propaganda, czyli według 
definicji Słownika Języka Polskiego ``technika sterowania poglądami i zachowaniami ludzi polegająca na 
celowym, natarczywym, połączonym z manipulacją oddziaływaniu na zbiorowość''~\cite{SJP}.
Pomimo iż najczęściej propaganda ma charakter polityczny nie jest to jedyne 
jej zastosowanie. Najstarszym przykładem pisemnej propagandy są opisy podbojów
Dariusza Wielkiego datowane na rok 515 p.n.e. Od tego czasu w historii ludzkości
można znaleźć wiele przypadków wykorzystania tego typu dezinformacji w krajach takich
jak Starożytny Rzym, Niemcy podczas II Wojny Światowej, a nawet w dzisiejszych 
czasach Korea Północna.
Propagandę można podzielić na 3 różne typy:
\begin{itemize}
    \item biała propaganda - źródło pochodzenia informacji jest prawdziwe i podane,
    \item szara propaganda - źródło pochodzenia informacji jest dla odbiorcy nieznane i może się on jedynie domyślać,
    \item czarna propaganda - źródło pochodzenia informacji jest umyślnie sfałszowane w celu wyrządzenia szkody.
\end{itemize}


% historia fake newsow 
\section{Sposoby rozprzestrzeniania fałszywych informacji}
Wraz ze zmianami w sposobach rozprzestrzeniania informacji na świecie zmieniało się
także podejście do tworzenia \textit{fake newsów} w odpowiedni sposób oszukujących osoby, do których 
były one skierowane. 

\subsection{Gazety}
Wykorzystanie \textit{fake newsów} w gazetach miało na celu przyciągnąć uwagę, a co za 
tym idzie zwiększyć sprzedaż danej gazety. Stało się to na tyle popularne, że spowodowało
narodziny nowego pojęcia ``żółtej prasy'', czyli takiej, której działanie polega na zamieszczaniu 
w nagłówkach w pełni lub częściowo nieprawdziwych informacji, aby przyciągnąć uwagę przechodnia 
za wszelką cenę, nawet jeśli wiąże się to z utratą wiarygodności. 

Dziennikarz Frank Luther Mott wyróżnia 5 cech charakteryzujących 
żółtą prasę ~\cite{YellowPressFrank}:
\begin{itemize}
    \item napisane dużą czcionką straszące nagłówki na temat mniej ważnych wydarzeń,
    \item nadmierna liczba zdjęć i rysunków,
    \item zawarcie sfałszowanych wywiadów, mylących nagłówków, pseudonauki oraz nieprawdziwych informacji od ludzi podających się za ekspertów,
    \item dodanie w pełni kolorowych dodatków do gazet w niedzielę,
    \item stawianie siebie jako słabszego w walce przeciwko systemowi.
\end{itemize}

\begin{figure}[h!]
    \centering
    \includegraphics[width=0.5\textwidth]{./Img/fake-newspaper.jpg}
    \caption{Przykład żółtej prasy z roku 1993 Źródło: https://www.nytimes.com/}
\end{figure}

\subsection{Telewizja}
Wynalezienie telewizji na początku XX wieku zmieniło całkowicie sposób, w jaki ludzie pozyskiwali wiadomości ze świata.
Aby pozyskać informacje na temat najnowszych wydarzeń, nie było konieczne kupienie gazety, a nawet 
wyjście z domu. W tym celu wystarczyło posiadać dostęp do telewizji i urządzenie do jej odbioru. Połączenie zarówno obrazu
jak i dźwięku zmusiło osoby chcące oszukać swoich odbiorów do stworzenia nowych technik 
pozwalających w wiarygodny sposób przedstawić kłamstwo.

Przykładem osoby, która w znakomity sposób wykorzystała siłę daną mu przez telewizję był Edward Bernays
nazywany ``Ojcem public relations''.
W roku 1929 został on zatrudniony aby wypromować papierosy firmy ``Lucky Strike'',
stworzona przez niego reklama ukazywała kobiety palące papierosy podczas marszu. 
Ponieważ kobiety palące były uznawane w tamtych czasach za temat
taboo, autor reklamy nazwał ją w gazetach walką o prawa kobiet. Reklama ta spowodowała tak duże 
spopularyzowanie palenia papierosów, że właśnie Edwardowi Bernays przypisuje się
ich dużą sprzedaż przez kolejne lata.
Był on także osobą dzięki której w dzisiejszych czasach diamenty są uznawane za symbol miłości po tym jak
został zatrudniony do wypromowania diamentów firmy ``De Beers''~\cite{MarkDice}.

\subsection{Internet}
Pojawienie się internetu wpłynęło na każdy element życia codziennego. Czynności takie jak komunikacja, rozrywka,
a także rozprzestrzenianie informacji uległy zmianom tak wielkim, że ciężko wyobrazić sobie w jaki sposób działały
one wcześniej. 

Dzięki pojawieniu się takich portali społecznościowych jak Facebook, Twitter oraz Instagram każdy użytkownik
internetu może opowiedzieć o swoich przemyśleniach lub wydarzeniach z życia każdej osobie zainteresowanej. Portale te
pozwoliły nie tylko na przekazywanie informacji o sobie, ale także opowiadanie o wydarzeniach ze świata przez wszystkie
chętne osoby. Wraz z rozwojem internetu stopniowo zwiększa się ilość osób czerpiących informacje na temat wydarzeń
z portali społecznościowych i do dnia dzisiejszego w Ameryce wynosi 68\% dorosłych osób. 

Udostępnienie każdej osobie możliwości wypowiedzenia się doprowadziło do sytuacji, w której duża część informacji w internecie
jest całkowicie fałszywa bądź w pewien sposób zmanipulowana poprzez osobę nieobiektywnie opisującą wydarzenia. Ogrom informacji
można zauważyć na podstawie ilości potwierdzonych \textit{fake newsów} związanych z wydarzeniami wokół wirusa COVID-19, których do 
czerwca 2020 jest aż 110. Niektóre z nich to~\cite{Korona}:
\begin{itemize}
    \item szczepionka na koronawirusa jest ukrywana od marca,
    \item aspiryna jest lekarstwem na COVID-19,
    \item komary przenoszą koronawirusa,
    \item kraje bez sieci 5G są wolne od koronawirusa ,
    \item koronawirus nie zagraża uczestnikom zgromadzeń religijnych,
    \item koronawirus to środek do zmniejszenia populacji Ziemi.
\end{itemize}
Osoby oszukane \textit{fake newsami} grają istotną rolę w dalszym ich rozprzestrzenianiu poprzez udostępnianie informacji swoim rozmówcom
m.in. znajomym, rodzinie. Rozprzestrzenianie informacji jest to cecha internetu, która daje niespotykaną wcześniej efektywność oszukiwania dużej ilości ludzi w szybkim
czasie. 
\begin{figure}[h!]
    \centering
    \includegraphics[width=0.7\textwidth]{./Img/cvfakenews.png}
    \caption{Przykład fałszywej informacji na portalu Facebook Źródło: https://www.facebook.com/}
\end{figure}

Na powyższym obrazie ukazany jest post udostępniony na portalu Facebook z którego wynika, że jeżeli osoba jest w stanie wstrzymać 
oddech na 10 sekund to nie jest ona zarażona koronawirusem. Treść postu wskazywała, że metoda ta według ekspertów 
z Japonii miała być skuteczna. Jak się później okazało informacja ta była całkowicie fałszywa. Nie powstrzymało to jednak ponad 2,4 tysiąca ludzi przed 
udostępnieniem jej swoim rozmówcom.~\cite{KoronaOddech} 

\section{Sposoby ochrony przed nieprawdziwymi informacjami}
Rozwój tak potężnego narzędzia jak internet spowodował, że fałszywe informacje
stały się poważnym zagrożeniem w dzisiejszym świecie. Aby zapobiec oszukaniu 
dużej ilości społeczeństwa znaleziono różne sposoby na ochronę przed nieprawdziwymi
informacjami, niektóre z nich to: 
\begin{itemize}
    \item IFLA ``How to spot fake news'' jest to stworzona przez międzynarodową instytucję
    reprezentującą interesy bibliotekarzy i pracowników informacji infografika
    przedstawiająca listę rzeczy, które należy zrobić podejrzewając, że jesteśmy oszukiwani.

    \begin{figure}[h!]
        \centering
        \includegraphics[width=0.5\textwidth]{./Img/how-to-spot-fake-news.jpg}
        \caption{Infografika stworzona przez IFLA Źródło: https://www.ifla.org/}
    \end{figure}

    Czynności które są na niej zawarte to: 
    \begin{itemize}
        \item sprawdzenie źródła informacji,
        \item dokładne przeczytanie treści,
        \item sprawdzenie autora,
        \item analiza odnośników,
        \item sprawdzenie dat związanych,
        \item upewnienie się, że informacja nie jest formą żartu,
        \item obiektywna ocena informacji,
        \item zapytanie ekspertów.
    \end{itemize}
    \item strony internetowe stworzone w celu walki z dezinformacją istnieje
    wiele witryn gdzie eksperci sprawdzają nadesłane przez użytkowników informacje
    pod względem ich zgodności z prawdą.
    Przykładami takich portali są: \emph{fakenews.pl, snopes.com, FactCheck.org, factchecker.in} 
    \item grupy osób powołanych przez rząd do walki z fałszywymi informacjami, najlepszym przykładem
    takiej grupy są tzw. \emph{Litewskie Elfy} - jest to grupa ochotników, którzy w wolnym czasie przeglądają 
    fora internetowe oraz media społecznościowe sprawdzając znajdujące się w nich informacje. W przypadku
    znalezienia nieprawdziwej informacji osoby te informują administratora strony o problemie, co najczęściej
    prowadzi do jego rozwiązania. Nazwa grupy wzięła się stąd, że ludzie ci walczą z 
    grupami powszechnie zwanymi \emph{trollami}. Grupa ta zyskała na popularności
    w takim stopniu, że rozpoczęto organizację kolejnych oddziałów w innych krajach
    między innymi na Łotwie.~\cite{Elves}
\end{itemize}
Pomimo istnienia wielu sposobów ochrony przed fałszywymi informacjami ich ilość powoduje,
że bardzo łatwo zostać oszukanym. Z tego powodu bardzo pomocne byłoby stworzenie oprogramowania
pozwalającego na automatyczne rozpoznanie czy informacja jest prawdziwa.
Implementacja takiego systemu w mediach społecznościowych jak Facebook lub Twitter pozwoliłaby na usunięcie 
kłamstw zanim trafiłyby one do użytkowników. Popularne w ostatnich czasach algorytmy uczenia
maszynowego \emph{ML} oraz analizy języka naturalnego \emph{NLP} osiągają zaskakująco dobrą 
poprawność w klasyfikacji różnego rodzaju tekstów, więc ich wykorzystanie w rozwiązaniu takiego 
problemu mogłoby być bardzo pomocne.

\section{Deep fake}

Data pojawienia się technologii deep fake nie jest dokładnie znana jednak zaczęła zyskiwać na popularności 
w grudniu 2017 roku, kiedy użytkownik o pseudonimie ``deepfakes'' umieścił na portalu \emph{Reddit} film
pornograficzny w którym główną rolę grała aktorka Gal Gadot znana z filmu ``Wonder Woman''. Aktorka jednak 
nigdy nie brała udziału w tego typu filmach a całe nagranie zostało stworzone wykorzystując algorytmy sztucznej
inteligencji, która pozwoliła na zamianę twarzy dowolnego aktora z nagrania twarzą Gal Gadot w bardzo realistyczny sposób.

Sytuacja ta przyciągnęła zainteresowanie wielu użytkowników i już w styczniu 2018 roku pojawiła się aplikacja o nazwie
``FakeApp'', dzięki której każdy może zamienić twarze znajdujące się na nagraniu na twarze kogoś innego.
Powszechny dostęp oraz prostota w użytkowaniu powodują, że filmy ``deep fake'' stanowią niespotykane wcześniej zagrożenie
mogąc oskarżyć dowolną osobę o wykonanie czynności, z którymi nie miała ona żadnego związku. 

Aby zwrócić uwagę na niebezpieczeństwo firma buzzfeed w kwietniu 2018 roku umieściła na swoim kanale \emph{Youtube} film, w którym Barack Obama opowiada 
o zagrożeniu płynącym z dezinformacji jednak słowa które wypowiada nie są naprawdę mówione przez niego, a pochodzą
z nagrania aktora Jordana Peele, które następnie zostało podrobione techniką ``deep fake''.

\begin{figure}[h!]
    \centering
    \includegraphics[width=0.7\textwidth]{./Img/peele.jpg}
    \caption{Klatka z filmu firmy Buzzfeed Źródło: https://www.youtube.com/}
\end{figure}

Filmy ``deep fake'' doprowadziły do sytuacji gdzie nie istnieje sposób przekazu informacji, który daje stu procentową wiarygodność,
co w jeszcze większym stopniu zwiększa znaczenie odnalezienia uniwersalnej i efektywnej metody walki z dezinformacją. 