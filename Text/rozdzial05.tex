\chapter{Projekt i implementacja systemu}
Do wykonania badań efektywności algorytmów uczenia maszynowego w rozpoznawaniu
fałszywych informacji potrzebne było stworzenie systemu, który w prosty sposób 
dla każdego z badanych algorytmów wykonałby uczenie go na danych treningowych 
a następnie sprawdził jak dobrze przewiduje on przypadki ze zbioru testowego.
System musi także w odpowiedni sposób przygotować dane tekstowe przy użyciu takich
metod jak zamiana dużych liter na małe oraz usuwanie znaków interpunkcyjnych itd.
Ważną funkcjonalnością oprogramowania jest też to by dzielił on dane tekstowe na 
różnej długości NGramy. 

NGramy są to sekwencje następujących po sobie jednostek, którymi mogą być słowa, 
głoski lub litery. W pracy wykonane zostaną badania różnych algorytmów klasyfikacji
na podstawie podziału na różnej długości NGramy literowe a następnie wyniki zostaną
porównane w celu odnalezienia optymalnych ustawień do rozpoznawania Fake newsów.
\section{Wykorzystane technologie}

\section{Wymagania funkcjonalne}

\section{Implementacja}

