\chapter{Podsumowanie}
Celem pracy było sprawdzenie czy algorytmy uczenia maszynowego mogą zostać wykorzystane
w detekcji tak zwanych \textit{fake newsów} i jak dobrze wykonają to zadanie. Cel pracy został osiągnięty,
a otrzymane wyniki jednoznacznie wskazują, że tego typu algorytmy bez większych problemów dobrze 
radzą sobie z klasyfikacją nieprawdziwych informacji.


W badaniach postanowiono sprawdzić metodę podziału tekstu na rożnej długości ngramy i zwrócić uwagę 
jak zmiana ich długości wpływa na poszczególne cechy algorytmów.
Wyniki badań wykonanych na algorytmach: \textit{k}-NN, RF, Naive Bayes,
SVC oraz MLP pokazują, że algorytmy uczenia z całą pewnością mogą znaleźć swoje miejsce w systemach
automatycznego rozpoznawania fałszywych informacji. Wiele z nich nie miało problemu z osiągnięciem 
efektywności powyżej 90\%. Najlepszym z badanych algorytmów okazał się RF osiągając
wynik 99.51\%. Najgorzej z zadaniem poradziła sobie metoda \textit{k}-NN, jej najwyższa 
poprawność wynosiła 80.07\%.

Jak przypuszczano, ogromne znaczenie dla efektywności
miała zmiana długości ngramów, które nie mogą być ani za krótkie, ani za długie.
Dla wszystkich badanych algorytmów za wyjątkiem \textit{k}-NN, ngramy o długości 1 dały najgorsze wyniki, a wartości
pośrednie najlepsze.

Istotnym problemem z tego typu rozwiązaniami jest brak zbiorów treningowych posiadających 
artykuły o zróżnicowanej tematyce. Większość istniejących danych zawiera głównie artykuły związane z 
polityką, z uwagi na to właśnie z taką tematyką algorytmy poradziłyby sobie najlepiej. Mogłyby mieć one jednak problem 
z artykułami naukowymi, sportowymi itp.

Jednym ze sposobów na dalszy rozwój pracy mogłoby być opracowanie systemu rozpoznającego 
przerobione obrazy. System taki w kombinacji z wykrywaniem fałszywych tekstów zawartym
w niniejszej pracy jeszcze bardziej zwiększyłby poprawność detekcji \textit{fake newsów} i stanowiłby 
duży krok w wyeliminowaniu z internetu zagrożenia jakim są nieprawdziwe informacje.