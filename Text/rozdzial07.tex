\chapter{Podsumowanie}
Tematem niniejszej pracy magisterskiej było 
``Zastosowanie metod uczenia maszynowego w detekcji fałszywych informacji'',
natomiast celem było sprawdzenie czy algorytmy uczenia maszynowego mogą zostać wykorzystane
w detekcji tak zwanych Fake newsów i jak dobrze wykonają to zadanie.


Postanowiono sprawdzić metodę podziału tekstu na rożnej długości NGramy i zwrócić uwagę 
jak zmiana ich długości wpływa na poszczególne cechy algorytmów.
Wyniki badań wykonanych na algorytmach: K najbliższych sąsiadów, Random Forest, Naive Bayes,
SVC oraz MLP pokazują, że algorytmy uczenia jak najbardziej mogą znaleźć swoje miejsce w systemach
automatycznego rozpoznawania fałszywych informacji. Wiele z nich nie miało problemu z osiągnieciem 
efektywności powyżej 90\%. Najlepszym z badanych algorytmów okazał się Random Forest osiągając
wynik 99.82\%. Najgorzej z zadaniem poradziła sobie metoda K najblizszych sąsiadów, jej najwyższa 
efektywność wynosiła 78.13\%.

Jak przypuszczano ogromne znaczenie dla efektywności
miała zmiana długości NGramów, które nie mogą być ani za krótkie, ani za długie.
Dla wszystkich badanych algorytmów poza KNN Ngramy o długości 1 dały najgorsze wyniki, a wartości
pośrednie najlepsze.

Istotnym problemem z tego typu rozwiązaniami jest brak zbiorów treningowych posiadających 
artykuły o zróżnicowanej tematyce. Większość istniejących danych zawiera głownie artykuły związane z 
polityką, przez co z taką tematyką algorytmy poradziłyby sobie najlepiej. Mogłyby mieć one jednak problem 
z artykułami naukowymi, sportowymi itp.

Jednym ze sposobów na dalszy rozwój pracy mogłoby być stworzenie systemu rozpoznającego 
przerobione obrazy. System taki w kombinacji z wykrywaniem fałszywych tekstów zawartym
w niniejszej pracy jeszcze bardziej zwiększyłby efektywność detekcji fake newsów i stanowiłby 
duży krok w wyeliminowaniu z internetu zagrożenia jakim są nieprawdziwe informacje.